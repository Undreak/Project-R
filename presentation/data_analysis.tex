\documentclass{beamer}

\usepackage[utf8]{inputenc}
\usepackage[T1]{fontenc}
\usepackage{graphicx}
\usepackage{listings}
\usepackage{xcolor}

\definecolor{codegreen}{rgb}{0,0.6,0}
\definecolor{codegray}{rgb}{0.5,0.5,0.5}
\definecolor{codepurple}{rgb}{0.58,0,0.82}
\definecolor{backcolour}{rgb}{0.95,0.95,0.92}

\lstdefinestyle{mystyle}{
    backgroundcolor=\color{backcolour},
    commentstyle=\color{codegreen},
    keywordstyle=\color{magenta},
    numberstyle=\tiny\color{codegray},
    stringstyle=\color{codepurple},
    basicstyle=\ttfamily\footnotesize,
    breakatwhitespace=false,
    breaklines=true,
    captionpos=b,
    keepspaces=true,
    numbers=left,
    numbersep=5pt,
    showspaces=false,
    showstringspaces=false,
    showtabs=false,
    tabsize=2
}

\lstset{style=mystyle}

\title{Statistical Data Analysis Project}
\author{Alexandre De Cuyper}
\date{February 22, 2024}
\institute{University of A Coruña}

\begin{document}

\frame{\titlepage}

\begin{frame}{Introduction}
    \begin{itemize}
        \item Brief overview of the data analysis project.
        \item Importance of the analysis: Understanding the shift in peak positions in crystals of different compositions.
        \item Objectives and hypotheses: Investigating if there are significant differences in peak positions among crystals.
    \end{itemize}
\end{frame}

\begin{frame}{Data Overview}
    \begin{itemize}
        \item Overview of the dataset: Peaks from different crystals with compositions and varying 2 theta values.
        \item Key variables and their significance: `2_theta`, Condition, Value, and Cluster.
    \end{itemize}
\end{frame}

\begin{frame}[fragile]{Data Preprocessing}
    \begin{itemize}
        \item Cleaning and handling missing values.
        \item Transformation of data for analysis.
    \end{itemize}

    \begin{lstlisting}[language=R]
    > head(data)
      2_theta Ni75Co25 Ni50Co50 Ni25Co75   Co
    1    8.00     20.0    18.00    24.00 19.0
    2    8.05     19.5    21.98    23.01 20.5
    3    8.10     19.0    26.00    22.00 22.0
    4    8.15     19.0    27.49    21.50 25.0
    5    8.20     19.0    29.00    21.00 28.0
    6    8.25     17.0    24.53    24.98 24.5
    \end{lstlisting}
\end{frame}

\begin{frame}[fragile]{Exploratory Data Analysis}
    \begin{itemize}
        \item Visualizations and summary statistics.
        \item Identification of patterns and trends.
    \end{itemize}

    \begin{lstlisting}[language=R, basicstyle=\tiny\ttfamily]
    # Exploratory Data Analysis (EDA)
    ggplot(data, aes(x = `2_theta`)) +
  geom_ribbon(aes(ymin = 0, ymax = `Ni25Co75`, fill = "Ni25Co75"),
    alpha = 0.1, color = "green"
  ) +
  geom_ribbon(aes(ymin = 0, ymax = `Ni50Co50`, fill = "Ni50Co50"),
    alpha = 0.1, color = "red"
  ) +
  geom_ribbon(aes(ymin = 0, ymax = `Ni75Co25`, fill = "Ni75Co25"),
    alpha = 0.1, color = "blue"
  ) +
  geom_ribbon(aes(ymin = 0, ymax = `Co`, fill = "Co"),
    alpha = 0.1, color = "yellow"
  ) +
  labs(
    title = "Powder X-ray Diffraction of Ni_(1-x)Co_x perovskites",
    x = "2 theta",
    y = "Intensity",
    fill = "Composition"
  ) +
  scale_fill_manual(values = c(
    "Ni75Co25" = "blue", "Ni50Co50" = "red",
    "Ni25Co75" = "green", "Co" = "yellow"
  )) +
  theme_minimal()
    \end{lstlisting}
\end{frame}

\begin{frame}{Visualization of the Data}
    \begin{figure}
        \centering
        \includegraphics[width=0.8\textwidth]{../plot/graph_nb.png}
        \caption{Powder X-ray Diffraction of Ni\_(1-x)Co\_x perovskites}
        \label{fig:xrd}
    \end{figure}
\end{frame}

\begin{frame}[fragile]{Exploratory Data Analysis}
    \begin{itemize}
        \item Visualizations and summary statistics.
        \item Identification of patterns and trends.
    \end{itemize}

    \begin{lstlisting}[language=R, basicstyle=\tiny\ttfamily]
      #Threshold value
        +
      geom_hline(
        yintercept = threshold,
        linetype = "dashed", color = "black"
      )
    \end{lstlisting}
\end{frame}

\begin{frame}{Visualization of the Data}
    \begin{figure}
        \centering
        \includegraphics[width=0.8\textwidth]{../plot/graph.png}
        \caption{Powder X-ray Diffraction of Ni\_(1-x)Co\_x perovskites without background and with threshold value}
        \label{fig:xrd}
    \end{figure}
\end{frame}




\begin{frame}[fragile]{Peak Identification}

    \begin{lstlisting}[language=R, basicstyle=\tiny\ttfamily]
    find_peaks <- function(data) {
      if (length(data) < 2) {
        return(NULL) # No peak in lists with 0 or 1 element
      }
      peaks <- rep(0, length(data))
      # Create a vector to store the value and index of the peaks
      for (i in 2:(length(data) - 1)) {
        if (data[i] > data[i - 1] && data[i] > data[i + 1]) { # Looking for a peak
          peaks[i] <- data[i]
        }
      }
      # Checking if the first and last value is a peak or not
      if (data[1] > data[2]) {
        peaks[1] <- data[1]
      }
      if (tail(data, 1) > tail(data, 2)[1]) {
        peaks[length(peaks)] <- data[length(data)]
      }
      return(peaks)
      }    \end{lstlisting}
\end{frame}

\begin{frame}{Peak Identification Results}
    \begin{figure}
        \includegraphics[width=1.1\textwidth]{../plot/peaks.png}
    \end{figure}
\end{frame}




\begin{frame}[fragile]{Clustering Process}
    \begin{itemize}
        \item Explanation of the clustering algorithm used.
        \item Description of the data preparation steps.
    \end{itemize}

    \begin{lstlisting}[language=R, basicstyle=\small\ttfamily]
    # Your clustering code here
    \end{lstlisting}
\end{frame}

\begin{frame}{Clustering Results}
    \begin{itemize}
        \item Overview of the clustering results.
        \item Interpretation of clusters and their characteristics.
    \end{itemize}

    \begin{figure}
        \includegraphics[width=0.7\textwidth]{cluster_plot.png}
        \caption{Clustering results visualized.}
    \end{figure}
\end{frame}






\begin{frame}[fragile]{ANOVA Testing}
    \begin{itemize}
        \item Analysis of Variance (ANOVA) to test hypotheses.
        \item Results and interpretation.
    \end{itemize}

    \begin{lstlisting}[language=R]
    # ANOVA Testing
    anova_result <- aov(`2_theta` ~ Condition * Cluster, data = data_long)
    summary(anova_result)
    \end{lstlisting}
\end{frame}

\begin{frame}{Results and Discussion}
    \begin{itemize}
        \item Interpretation of ANOVA results.
        \item Comparison of clusters: Assessing the significance of peak position variations.
        \item Implications of the findings: Understanding how composition affects peak positions.
    \end{itemize}
\end{frame}

\begin{frame}{Conclusion}
    \begin{itemize}
        \item Summary of key findings.
        \item Limitations and areas for future research.
    \end{itemize}
\end{frame}

\begin{frame}{Questions \& Discussion}
    \centering
    \Huge Any Questions?
\end{frame}

\end{document}

